\section*{Sección 2}
\frame
{
\frametitle{}
\begin{center}
\begin{huge}
\hspace*{1cm}Conceptos asociados\newline\newline
\end{huge}
\includegraphics[width=0.5\textwidth]{imgs/conceptos.jpg}
\end{center}
}

\section{Conceptos asociados}
\frame
{
\frametitle{Introducción de conceptos}
\begin{itemize}
 \item Importante familiarizarse con algunos términos del mundillo.
 \item La razón es conocer de forma general qué papel juegan ciertos elementos en los procesadores.
 \item Una visión clara de los conceptos asociados al tema nos facilita la elección de un procesador.
\end{itemize}
}

\frame
{
\frametitle{Socket de CPU}
\begin{itemize}
 \item El \textbf{socket de CPU} es una matriz de pequeños agujeros (zócalo) existente en una placa base donde encajan los pines de un microprocesador; dicha matriz, denominada \textit{Pin grid array} o PGA, permite la conexión entre el microprocesador y dicha placa base.
 \item Ejemplos de socket de CPU son: Socket 939 (AMD), Socket AM2 (AMD), Socket 478 (Intel), Socket 775 (Intel)...\vspace*{0.5cm}

\hspace*{3cm}
\includegraphics[width=1.5cm]{imgs/socket939.jpg}
\hspace*{1cm}
\includegraphics[width=3.5cm]{imgs/socket775.jpg}

\end{itemize}
}

\frame
{
\frametitle{Niveles de caché}
\begin{itemize}
 \item \textbf{Propósito de la caché:} actuar como una memoria temporal entre los registros de CPU, limitados y de gran velocidad y la RAM, mucho más grande y lenta.
 \item Los subsistemas de caché pueden ser de \textbf{niveles múltiples}; es decir, puede haber más de un conjunto de caché entre el CPU y la memoria principal. 
 \item Muchos sistemas tienen \textbf{dos niveles de caché}:
	\begin{itemize}
	\item \underline{Caché L1}  $\Rightarrow$ ubicada en el chip de la CPU, se ejecuta a la misma velocidad que dicha CPU.
	\item \underline{Caché L2} $\Rightarrow$ suele ser parte del módulo de la CPU, se ejecuta a las mismas velocidades que la CPU (o casi) y es un poco más grande y lenta que la caché L1.
	\end{itemize}
 \item Algunos sistemas (normalmente servidores) también tienen \textbf{caché L3} formando parte del sistema de la placa base. La caché L3 es más grande y  algo más lenta que la caché L2.
\end{itemize}
}

\frame
{
\frametitle{MMX}
\begin{itemize}
 \item Es el acrónimo de \textbf{M}ulti\textbf{m}edia E\textbf{x}tensions.
 \item Conjunto de instrucciones \textbf{SIMD} (Single Instruction Multiple Data) diseñado por Intel e introducido en 1997 en sus microprocesadores Pentium MMX.
 \item MMX agregó \textbf{8 nuevos registros} a la arquitectura, conocida como MM0 a MM7. En realidad, estos nuevos registros son meros alias de los registros de la pila de la FPU. Cada uno de los registros MMn es un número entero de 64 bits.
 \item El juego de instrucciones MMX utiliza el concepto del \textbf{tipo de datos compactados} $\Rightarrow$ en lugar de usar el registro completo para un solo número entero de 64 bits, se usa para almacenar dos enteros de 32 bits, cuatro enteros de 16 bits u ocho enteros de 8 bits.
 \item \textbf{Problema:} MMX sólo soporta operaciones con números enteros. Hace algún tiempo, el uso de matemáticas de vector entero tenía sentido (operaciones 2D y 3D), pero cuando esta funcionalidad se pasa a las GPUs, la coma flotante se vuelve mucho más importante.
 \end{itemize}
}

\frame
{
\frametitle{SSE}
\begin{itemize}
 \item \textbf{SSE} (Streaming SIMD Extensions) es una extensión al grupo de instrucciones MMX.
 \item Estas instrucciones operan con paquetes de operandos en \textbf{coma flotante} de precisión simple.
 \item Hay varios tipos de instrucciones SSE:
	\begin{itemize}
		\item Instrucciones SSE de Transferencia de datos.
 		\item Instrucciones SSE de Conversión.
 		\item Instrucciones SSE Aritméticas.
 		\item Instrucciones SSE lógicas.
	\end{itemize}
 \item Con la tecnología SSE, se introducen 70 nuevas instrucciones y 8 registros nuevos: del xmm0 al xmm7.
 \item Los registros tienen una extensión de 128 bits. A diferencia de MMX, la utilización de SSE no implicaba la inhabilitación de la FPU, por lo que no era necesario habilitarla nuevamente, lo que significaba para MMX pérdida de velocidad.
\end{itemize}
}

\frame
{
\frametitle{FSB (Front Side Bus)}
\begin{itemize}
 \item La CPU está conectada a \textbf{un bus} que le permite comunicarse con el resto de dispositivos.
 \item Gracias a este bus frontal de datos, llamado \textbf{FSB} (Front Side Bus), la CPU recibe información y la envía a otros dispositivos.
 \item El FSB se encuentra conectado al chip \textbf{Northbridge}, que es el núcleo de la placa base.
 \item La frecuencia de un procesador se expresa en términos de la \textbf{frecuencia del FSB} multiplicado por un valor predeterminado por el fabricante, por eso conocer bien el FSB es vital en la práctica del Overclocking (forzar un procesador a trabajar a una velocidad mayor que la de serie).
 \item \textbf{Ejemplo} $\Rightarrow$ \textbf{Multiplicador:} x18, \textbf{Frecuencia del FSB:} 200MHz, \textbf{Frecuencia del procesador:} 3600 MHz.
\end{itemize}
}

\frame
{
\frametitle{FSB (Front Side Bus)}
\begin{center}
\includegraphics[width=0.6\textwidth]{imgs/nsbridge.png}
\end{center}
}
