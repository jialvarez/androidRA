\section*{Sección 6}
\frame
{
\frametitle{}
\begin{center}
\begin{huge}Comparando los distintos procesadores\newline\newline
\end{huge}
\includegraphics[width=0.3\textwidth]{imgs/comparison1.png}\hspace*{1cm}
\includegraphics[width=0.3\textwidth]{imgs/comparison2.jpg}
\end{center}
}

\section{Comparando los distintos procesadores}
\frame
{
\frametitle{Cómo vamos a realizar la comparación}
\begin{itemize}
 \item Vamos a analizar las especificaciones de los procesadores de las compañías líderes mediante unas tablas de datos.
 \item Nos centramos en el hecho de que un procesador teóricamente idéntico que otro con el mismo nombre clave es inferior debido a que difieren en el número de procesador.
 \item Cada número de procesador nos marca unas características.
 \item Cada compañía tiene un \textit{sitio Web} con utilidades de comparación de sus procesadores.
	\begin{itemize}
	\item \textbf{Intel} $\Rightarrow$ http://compare.intel.com
	\item \textbf{AMD} $\Rightarrow$ http://www.amdcompare.com
	\end{itemize}
 \item Diferenciamos entre equipos de sobremesa (escritorio) y equipos portátiles. Además, dividimos por compañía.
\end{itemize}
}

\frame
{
\frametitle{Procesadores Intel de escritorio}
\begin{itemize}
\item Consideraremos los siguientes procesadores Intel de escritorio:
	\begin{itemize}
	\item \textbf{Pentium D:} dos procesadores Pentium 4 (de núcleo \textit{Prescott}) \textbf{sin} HyperThreading con pequeñas mejoras internas, metidos ambos en una única pieza de silicio.
	\item \textbf{Pentium Extreme Edition:} no confundir con el Pentium 4 Extreme Edition, el Pentium \textbf{Extreme Edition} contiene dos procesadores \textit{Pentium 4 Prescott}, \textbf{con} tecnología Hyperthreading.
	\item \textbf{Pentium Dual Core:} basados en el procesador mononúcleo \textit{Conroe-L}, que no era suficiente para distinguir entre las marcas \textit{Pentium} y \textit{Celeron}, por lo que se sustituyó por CPUs de doble núcleo.
	\item \textbf{Intel Core 2 Duo:} la continuación de los \textit{Pentium D} y \textit{Core Duo} (éste último lanzado en enero de 2006). Nombre clave: \textit{Conroe}.
	\item \textbf{Intel Core 2 Quad:} procesadores con 4 núcleos y de 64 bits, un 70\% más rápidos que los \textit{Core 2 Duo}.
	\item \textbf{Intel Core 2 Extreme:} tienen multiplicador desbloqueado (hasta 40x), y se utilizan los mejores cristales en su fabricación, con lo cual el proceso de overclocking es más sencillo y tiene un potencial más alto.
	\end{itemize}
\end{itemize}
}

\frame
{
\frametitle{Procesadores Intel de escritorio}
\begin{itemize}
 \item \textbf{Tabla de especificaciones: procesador \textbf{Pentium D}}
\begin{center}
\includegraphics[width=0.9\textwidth]{imgs/tabla-pentium-d.png}
\end{center}
\end{itemize}
}

\frame
{
\frametitle{Procesadores Intel de escritorio}
\begin{itemize}
 \item \textbf{Tabla de especificaciones: procesadores \textbf{Pentium Dual Core y Extreme Edition}}
\begin{center}
\includegraphics[width=0.9\textwidth]{imgs/tabla-pentium-dual-y-extreme.png}
\end{center}
\end{itemize}
}

\frame
{
\frametitle{Procesadores Intel de escritorio}
\begin{itemize}
 \item \textbf{Tabla de especificaciones: procesador \textbf{Core 2 Duo}}
\begin{center}
\includegraphics[width=0.9\textwidth]{imgs/tabla-core2-duo.png}
\end{center}
\end{itemize}
}

\frame
{
\frametitle{Procesadores Intel de escritorio}
\begin{itemize}
 \item \textbf{Tabla de especificaciones: procesadores \textbf{Core 2 Quad y Core 2 Extreme}}
\begin{center}
\includegraphics[width=0.9\textwidth]{imgs/tabla-core2-quad-y-extreme.png}
\end{center}
\end{itemize}
}


\frame
{
\frametitle{Procesadores AMD de escritorio}
\begin{itemize}
\item Consideraremos los siguientes procesadores AMD de escritorio:
	\begin{itemize}
 	\item \textbf{AMD Athlon 64 X2 Dual Core:} microprocesador de 64 bits de doble núcleo introducido para el socket 939 (en 90 nm) y para el socket AM2 (en 90 nm y 65 nm) con un bus HyperTransport de 2000 Mhz y soporte de memoria DDR2 a partir de los modelos AM2, y conjunto de instrucciones SSE3. Cada núcleo cuenta con una unidad de cache independiente.
	\end{itemize}

\item Se han desestimado para el estudio los siguientes procesadores:
	\begin{itemize}
	 \item \textbf{AMD Sempron:} procesador mononúcleo.
	 \item \textbf{AMD Athlon 64:} procesador mononúcleo.
	 \item \textbf{AMD Athlon 64 FX:} procesador mononúcleo destinado principalmente al disfrute de juegos y multimedia.
	 \item \textbf{AMD Athlon X2 Dual Core:} son sólo tres modelos que salieron bajo dicho sobrenombre y que fueron un impulso cualitativo para los reales AMD Athlon 64 X2 Dual Core.
	\end{itemize}
\end{itemize}
}

\frame
{
\frametitle{Procesadores AMD de escritorio}
\begin{itemize}
 \item \textbf{Tabla de especificaciones: procesador \textbf{AMD Athlon 64 X2 Dual Core}}
\begin{center}
\includegraphics[width=0.9\textwidth]{imgs/tabla-amd-athlon64x2.png}
\end{center}
\end{itemize}
}

\frame
{
\frametitle{Procesadores Intel para portátiles}
\begin{itemize}
 \item Por la cantidad de procesadores existentes, aquí vamos a comparar las tecnologías \textit{Centrino}, \textit{Centrino Duo} y \textit{Centrino Pro}.
 \item \textbf{Tabla de especificaciones: procesadores \textbf{Core Solo (1 núcleo), Core 2 Solo (1 núcleo), Core Duo y Core 2 Duo}}:
\begin{center}
\includegraphics[width=0.8\textwidth]{imgs/tabla-centrinos.png}
\end{center}
\end{itemize}
}

\frame
{
\frametitle{Procesadores AMD para portátiles}
\begin{itemize}
\item Consideraremos los siguientes procesadores AMD para portátiles:
	\begin{itemize}
	\item AMD Athlon 64 X2 Dual-Core.
	\item AMD Turion 64 X2 Dual-Core.
	\end{itemize}

 \item \textbf{Tabla de especificaciones: procesadores AMD Athlon 64 X2 Dual Core y AMD Turion 64 X2 Dual-Core}
\end{itemize}

\begin{center}
\includegraphics[width=0.8\textwidth]{imgs/tabla-amd-turion-athlon.png}
\end{center}
}
