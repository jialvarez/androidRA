\subsection*{Wikitude}
\frame
{
\frametitle{Wikitude}
\begin{itemize}
 \item Lanzamiento inicial en 2008 en Austria por la empresa Wikitude Gmbh
 \item Se estructura en \textit{targets}
 \item También proporciona acceso a su propia nube privada\\
   \url{https://www.layar.com/creator/}
 \item Disponible para Android, iOS, BlackBerry, Windows Phone, Phonegap y Titanium
 \item Ganador del premio \textit{Best Augmented Reality Browser, Augmented Planet} en 2009, 2010, 2011 y 2012, entre muchos otros
\end{itemize}
}

\frame
{
\frametitle{Wikitude: ventajas e inconvenientes}
\begin{itemize}
\item \textbf{Ventajas:}
  \begin{itemize}
   \item Documentación muy completa
   \item Más barato que Metaio y Layar (600\euro), incluyendo geolocalización
   \item Versión educacional con marca de agua a 0\euro
   \item Posibilidad de reconocimiento en la nube
   \item Web perfectamente preparada para la creación de campañas
   \item Soporte muy orientado a comunidad
  \end{itemize}

\item \textbf{Inconvenientes:}
  \begin{itemize}
   \item No es libre
   \item No permite montar un servidor de recursos propios
  \end{itemize}

\end{itemize}
}
