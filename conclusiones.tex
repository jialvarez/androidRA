\section{Mi elección}
\frame
{
\frametitle{Mi elección personal}
\begin{itemize}
 \item \textbf{Vuforia} es buena herramienta \textbf{gratis total} para desarrollar una aplicación de realidad aumentada con reconocimiento de imágenes
 \item Sin embargo, la parte de \textbf{geolocalización} habría que desarrollarla manualmente, y el tándem \textbf{JNI + NDK} es más engorroso para desarrollar
 \item \textbf{Wikitude} tiene una versión \textbf{Edu} gratuita con marca de agua, y es realmente sencillo, casi todo se hace con \textbf{Javascript}
 \item Para aplicaciones comerciales de peso, la inversión de Wikitude es de 600\euro \hspace{0.15cm}en un \textbf{único pago} y de 9\euro/mes por el uso de 3 imágenes en su nube. Si usamos la herramienta Target Manager nos sale \textbf{gratis}, pero desarrollaremos la parte Javascript manualmente
\end{itemize}
}
