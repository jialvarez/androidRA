\section*{Sección 4}
\frame
{
\frametitle{}
\begin{center}
\begin{huge}1 núcleo, 2 núcleos, 4 núcleos... Un poco \hspace*{1.5cm}de historia.\newline\newline
\end{huge}
\includegraphics[width=0.25\textwidth]{imgs/amd-opteron.jpg}\hspace*{1cm}
\includegraphics[width=0.28\textwidth]{imgs/intel-quad.jpg}
\end{center}
}

\section{1 núcleo, 2 núcleos, 4 núcleos... Un poco de historia.}
\frame
{
\frametitle{¿Necesitamos tanta capacidad?}
\begin{itemize}
\item La enferma carrera que mantienen Intel y AMD por superar al rival nos lleva a ver morir productos que ni siquiera pudimos consumir ni necesitábamos hacerlo.
\item Muchos ni siquiera disponemos aún de un procesador de doble núcleo, ni en el PC de escritorio ni en el portátil.
\item Es posible saltarse hasta una generación de procesadores en la compra de nuestro próximo equipo.
\end{itemize}
}

\frame
{
\frametitle{AMD vs. Intel y una carrera que no para}
\begin{itemize}
\item AMD lanzó sus procesadores doble núcleo, los Athlon64 X2, luego Intel hizo lo propio con su línea Pentium D.
\item Pentium D utilizaba la tecnología NetBurst, con cuello de botella para los datos y no alcanzaba en rendimiento a Athlon 64 X2.
\item Intel contraataca con los Core Duo, con nueva tecnología y diseño de 65 nanómetros. Más tarde, actualiza la tecnología a la actual \textbf{Core 2 Duo}.
\end{itemize}
}

\frame
{
\frametitle{Pero, ¿qué es doble núcleo?}
\begin{itemize}
\item Esta pregunta tan manida significa que la CPU, tiene no un procesador, si no dos en el mismo paquete y se distribuyen el trabajo.
\item Lo lógico es pensar que al tener dos cerebros se puede procesar el doble de información, pero, lamentablemente, no siempre es así.
\item Dos factores justifican esta limitación:

	\begin{itemize}
	\item[1)] \textbf{Ancho de banda / cuello de botella}
		\begin{itemize}
		\item Problema común en los Pentium D, que \textbf{comparten el FSB} para recibir información y devolverla procesada.
		\item El FSB está \textbf{limitado en ancho} y encima es compartido por ambos núcleos, por lo que los datos deben esperar su turno para procesarse.
		\item \textbf{AMD} creó la interconexión \textbf{HyperTransport}, que interconecta los núcleos en varias direcciones, lo que proporciona un canal directo entre el procesador y la memoria sin tener que compartirlo con nadie. 
		\item Problema de AMD $\Rightarrow$ cuando AMD estaba utilizando un \textbf{método de fabricación} de 130nm, Intel pasó al de 90nm, cuando AMD al fin pudo llegar a 90nm Intel se volvió a adelantar con el de 65nm.
		\item La ventaja de poder incluir \textbf{más en menos espacio} y que las conexiones y distancias de los circuitos sean más pequeñas es que se necesita menos energía para mover un electrón de un lugar al otro.
		\end{itemize}
	\end{itemize}
\end{itemize}
}

\frame
{
\frametitle{Pero, ¿qué es doble núcleo?}
\begin{itemize}
 	\item Dos factores justifican esta limitación:

	\begin{itemize}
	\item[2)] \textbf{Aplicaciones}

		\begin{itemize}
		\item \textbf{Pocas aplicaciones preparadas} para sacar provecho de dos núcleos (incluidos los juegos).
		\item El único lugar donde se saca realmente provecho es del \textbf{lado servidor} y procesamiento de video.
		\item Gran ventaja con múltiples núcleos $\Rightarrow$ renderizando una \textbf{imagen 3D de alta resolución}, cada núcleo se puede encargar de un frame, tener muchos núcleos nos multiplicaría el tiempo ahorrado.
		\item Por esta razón se utilizan \textbf{granjas de servidores} para procesar películas.
		\item Básicamente se aprovechan las ventajas en todas las tareas que se puedan dividir en hilos y no ser todo un conjunto de procesamiento.
		\end{itemize}

	\end{itemize}

\end{itemize}
}

\frame
{
\frametitle{Quad core: 4 núcleos efectivos.}
\begin{itemize}
\item \textbf{AMD: pionera} con su AMD Quad FX (AMD 4x4 antes de su lanzamiento).
\item Emplea \textbf{dos zócalos AM2} con HyperTransport, cada uno de los cuales permite una \textbf{CPU de doble núcleo} y un banco de \textbf{memoria DDR2}.
\item \textbf{Intel} contraataca con dos Core 2 Duo en un mismo paquete compartiendo el bus de datos a la memoria, llamándolos:
	\begin{itemize}
	\item \textbf{Core 2 Quad:} procesadores con 4 núcleos y de 64 bits. Son un 70\% más rápidos que los Core 2 Duo.
	\item \textbf{Core 2 Extreme:} tienen multiplicador desbloqueado (hasta 40x), y se utilizan los mejores cristales en su fabricación, con lo cual el proceso de overclocking es más sencillo y tiene un potencial más alto.
	\end{itemize}
\item \textbf{Para portátiles:} en el primer semestre de 2008 se actualizan los denominados \textbf{Intel Santa Rosa} con la tecnología \textbf{Core 2 Quad}. Los procesadores serán los llamados \textbf{\textit{Penryn}}.
\end{itemize}
}

\frame
{
\frametitle{Quad core: 4 núcleos efectivos.}
\begin{itemize}
 \item La ``competición'' no termina aquí: mientras Intel vende microprocesadores de cuatro núcleos que son \textbf{dos paquetes de dos núcleos cada uno}, AMD lanza los Opteron (nombre clave \textbf{\textit{Barcelona}}), con cuatro núcleos de verdad individuales dentro del propio procesador.
 \item El mercado de servidores se encuentra ahora con la dualidad AMD Opteron - Intel Xeon (Core 2 Extreme), ambos con 4 núcleos.
 \item Por si fuera poco, AMD lanza \textbf{\textit{Phenom}} para equipos de sobremesa, que llegan al mercado en el primer trimestre de 2008. Las versiones de triple núcleo (nombre código ``Toliman'') formarán las series \textit{Phenom 8000}, las versiones de cuatro núcleos (nombre código ``Agena'') formarán las series \textit{Phenom 9000}, y las versiones de gama alta (nombre código ``Agena FX'') serán las series \textit{Phenom FX}.
 \item No consiguen derrotar a \textbf{\textit{Intel Core 2 Quad}} ni siquiera en la que se suponía su mayor baza (consumo energético) ni en escala de integración (Intel utiliza ya ¡45nm!).

\end{itemize}
}

\frame
{
\frametitle{Curiosidades...}
\begin{itemize}
 \item Para identificar la información de un procesador:
\end{itemize}

\begin{center}
 \includegraphics[width=0.8\textwidth]{imgs/howto-procesador.png}
\end{center}
}

\frame
{
\frametitle{Curiosidades...}
\textbf{Shrek Tercero} se diseñó con el siguiente hardware:
\hspace*{2cm} \includegraphics[width=1cm]{imgs/shrek3.jpg}
\begin{itemize}
\item Servidores \textbf{HP ProLiant DL145} compuestos por procesadores \textbf{AMD Opteron} de doble núcleo y \textbf{8GB de RAM.}
\item Estaciones de trabajo \textbf{HP xw9300} compuestas de igual manera por procesadores \textbf{AMD Opteron} de doble núcleo.
\item Portátiles \textbf{HP nx6125} basadas en el procesador \textbf{AMD Turion64 X2}.
\item Para elaboración de la película se utilizaron la cantidad de 4000 núcleos es decir 2000 procesadores.
\item En 2001, \textbf{Shrek I} necesitó 5 millones de horas de CPU. En 2004, \textbf{Shrek 2 } precisó 10 millones, y en 2007 \textbf{Shrek 3} preciśo 20 millones.
\item El almacenamiento de Shrek 3 precisa 24 TB.
\end{itemize}
	\scriptsize \fbox{
	\begin{minipage}{11cm}
	Linux Red Hat Enterprise 4 como SO y Python para escribir las utilidades software.
	\end{minipage}
	}
}
