\subsection*{OpenCV}
\frame
{
\frametitle{OpenCV for Android}
\begin{itemize}
 \item Biblioteca libre de \textbf{visión artificial} originalmente desarrollada por \textbf{Intel}
 \item En 2008, la empresa \textbf{Willow Garage} asume el soporte. En 2012, lo hace la empresa \textbf{ItSeez}.
 \item Disponible para Windows, Linux, Mac, Android e iOS
 \item Para Android se proporciona la API Java con clases específicas, que es un subconjunto de la API de C
 \item SDK Quick start\\ \url{http://docs.opencv.org/doc/tutorials/introduction/android_binary_package/O4A_SDK.html}
 \item Utilizado en aeronaves no tripuladas, sistemas de vigilancia, reconocimiento facial, etc.
\end{itemize}
}

\frame
{
\frametitle{OpenCV for Android: ventajas e inconvenientes}
\begin{itemize}
\item \textbf{Ventajas:}
  \begin{itemize}
   \item Licencia BSD
   \item Buen rendimiento
   \item Multiplataforma
   \item Soporte de la comunidad. Multitud de snippets.
  \end{itemize}

\item \textbf{Inconvenientes:}
  \begin{itemize}
   \item La API de Java es un subconjunto mínimo. Para obtener un conjunto mayor, se recomienda usar el NDK + JNI. Más info: \\
     \url{http://www.nacho-alvarez.es/index.php/blog/2012/05/02/conectar-programas-cc-con-aplicaciones-android/}
   \item El sobreimpresionado de elementos debe hacerse manualmente
   \item Se centra en visión por computador, así que no tenemos la parte GPS
   \item Hace falta una formación específica en visión artificial para utilizarla correctamente
  \end{itemize}

\end{itemize}
}

\frame
{
\frametitle{OpenCV for Android: recursos}
\begin{itemize}
\item \textbf{OpenCV4Android:} \url{http://opencv.org/platforms/android.html}
\item \textbf{Quick Start:} \url{http://docs.opencv.org/doc/tutorials/introduction/android_binary_package/O4A_SDK.html}
\item \textbf{Android development with OpenCV:} \url{http://docs.opencv.org/doc/tutorials/introduction/android_binary_package/dev_with_OCV_on_Android.html}
\item \textbf{Java API:} \url{http://docs.opencv.org/java/}
\end{itemize}
}
