\section*{Sección 5}
\frame
{
\frametitle{}
\begin{center}
\begin{huge}
\hspace*{1cm}¿Y qué hay de los portátiles?\newline
\end{huge}
\includegraphics[width=0.8\textwidth]{imgs/portatiles.png}
\end{center}
}

\section{¿Y qué hay de los portátiles?}
\frame
{
\frametitle{Los procesadores móviles de Intel}
\begin{itemize}
 \item Ni mucho menos el avance en el diseño de \textbf{procesadores para portátiles} se ha quedado estancado.
 \item Intel ofrece las tecnologías \textbf{Centrino} y \textbf{Centrino Duo}.
 \item Son \textbf{tecnologías} desarrolladas para promocionar en el diseño de un ordenador portátil una combinación determinada de:

	\begin{itemize}
	\item CPU Intel Pentium M o, posteriormente, \textbf{Intel Core} o \textbf{Intel Core 2}.
	\item Chipset de la placa base familia Intel \textbf{855}, \textbf{915} o \textbf{945}.
	\item Interface de red inalámbrica del tipo Intel PRO/Wireless \textbf{2100 (IEEE 802.11a/b)} o PRO/Wireless 2200 \textbf{(IEEE 802.11b/g)} o posterior.
	\end{itemize}

 \item No se debe confundir al procesador Pentium M como ``el procesador Centrino'', ya que Centrino es \textbf{la tecnología} que engloba al procesador, al chipset y a la tarjeta de red inalámbrica Wi-Fi integrada.

\end{itemize}
}

\frame
{
\frametitle{Los procesadores móviles de Intel}
\begin{itemize}
 \item Intel diseñó su estrategia en base a una serie de \textbf{plataformas}:
	\begin{itemize}
		\item \textbf{\underline{Plataforma Carmel}}\\
		Plataforma original Centrino, lanzada en 2003. Consta de:
		\begin{itemize}
		\item CPU Pentium-M (nombre clave \textit{Banias}) bus 400 MHz, 1MB Caché L2.
		\item Chipset serie 855.
		\item Chip WiFi Intel PRO/Wireless 2100 o 2200.
		\end{itemize}
	
		\item \textbf{\underline{Plataforma Sonoma}}\\
		Plataforma que actualiza la original con la nueva generación de Centrino, lanzada en 2005. Consta de:
		\begin{itemize}
		\item CPU Pentium-M (algunos incluyen el núcleo mejorado con nombre clave \textit{Dohan}) bus 533 MHz, 2MB Caché L2.
		\item Chipset serie 915.
		\item Tecnología PCI Express.
		\item Chip WiFi Intel PRO/Wireless 2915 (IEEE 802.11a/b/g).
		\end{itemize}	
	\end{itemize}
\end{itemize}
}

\frame
{
\frametitle{Los procesadores móviles de Intel}
\begin{itemize}
 \item Intel diseñó su estrategia en base a una serie de \textbf{plataformas}:
	\begin{itemize}
	 \item \textbf{\underline{Plataforma Napa}}\\
	Versión de Centrino lanzada en 2006. Consta de:
		\begin{itemize}
		\item CPU Core Solo (Duo mononúcleo), Core Duo (nombre clave \textit{Yonah}) o posteriormente Core 2 Duo (\textit{Merom}). Las versiones de la plataforma Centrino basadas en CPU \textbf{Core Duo} y \textbf{Core 2 Duo} reciben el nombre de \textbf{Centrino Duo}.
		\item Chipset serie 945, que puede incluir gráficos integrados GMA950.
		\item Intel PRO/Wireless 3945 IEEE 802.11 a/b/g.
		\end{itemize}
	\end{itemize}

	\begin{itemize}
	 \item \textbf{\underline{Plataforma Santa Rosa} * Plataforma vigente en la actualidad *}\\
	Es la cuarta generación de la plataforma Centrino. Presentado el 9 de mayo de 2007, con:
		\begin{itemize}
		\item CPU Core 2 Duo (Merom 2ª generación).
		\item Chipset serie 965 (con gráficas integradas X3000, nombre clave \textit{Crestiline}).
		\item Intel PRO/Wireless 4965AGN IEEE 802.11 a/b/g/n.
		\end{itemize}
	\end{itemize}
\end{itemize}
}

\frame
{
\frametitle{Los procesadores móviles de Intel}
\begin{itemize}
 \item Intel diseñó su estrategia en base a una serie de \textbf{plataformas}:
	\begin{itemize}
	 \item \textbf{\underline{Plataforma Santa Rosa} * Plataforma vigente en la actualidad *}\\
	Analizando algo más en detalle:
		\begin{itemize}
		\item Se comercializan con los nombres de \textbf{Centrino Duo} (como los anteriores) y \textbf{Centrino Pro}.
		\item Se incluyen nuevos modelos de procesadores de 65 nm: los Core 2 T7x00, con 4 MB de caché L2 y FSB a 800 MHz.
		\item Incorporan la tecnología \textbf{Turbo Memory}, que sirve para emplear una memoria flash a modo de caché del disco duro para \textbf{aumentar el rendimiento} y \textbf{reducir el consumo}.
		\item Opinión personal: realmente rápido utilizando \textbf{Ubuntu}, compilando, instalando paquetes, etc...
		\end{itemize}
	\end{itemize}
\end{itemize}
}

\frame
{
\frametitle{Los procesadores móviles de Intel}
\begin{itemize}
 \item Intel diseñó su estrategia en base a una serie de \textbf{plataformas}:
	\begin{itemize}
	 \item \textbf{\underline{Plataforma Montevina}}\\
	El nombre código \textit{Montevina} se refiere a la quinta generación de la plataforma Centrino. Esta prevista para lanzarse a inicios del 2008. Montevina soportará:

		\begin{itemize}
		\item Procesador de ¡45nm! \textbf{Penryn} (4 núcleos).
		\item Chipset \textbf{Cantiga}, con FSB a \textbf{1GHz}.
		\item El módulo inalámbrico \textbf{Shiloh}, con soporte para \textbf{WiMAX} y \textbf{HSDPA} (optimización de UTMS, se le reconoce como 3.5G), además del controlador \textbf{LAN Boaz}.
		\item Memorias DDR3 (por confirmar).
		\end{itemize}
	\end{itemize}
\end{itemize}
}

\frame
{
\frametitle{Los procesadores móviles de AMD}
\begin{itemize}
 \item AMD basa su estrategia comercial para portátiles en \textbf{tres familias de procesadores}:

	\begin{itemize}
		\item \textbf{\underline{Mobile AMD Sempron}}\\
		Microprocesador de bajo coste con arquitectura X86 que se equipara al procesador Celeron de Intel. Las primeras versiones fueron lanzadas al mercado en agosto de 2004.
		\item \textbf{\underline{AMD \underline{Athlon} 64 X2 Dual-Core}}
		\begin{itemize}
			\item Microprocesador de 64 bits y doble núcleo. Consta de:
			\item Versiones para el Socket 939 (en 90 nm) y para el socket AM2 (en 90 nm y 65 nm).
			\item Bus HyperTransport de 2000 Mhz.
			\item Soporte de memoria DDR2 a partir de los modelos AM2 (Julio 2006) y conjunto de instrucciones SSE3.
		\end{itemize}
	\end{itemize}
\end{itemize}
}

\frame
{
\frametitle{Los procesadores móviles de AMD}
\begin{itemize}
 \item AMD basa su estrategia comercial para portátiles en \textbf{tres familias de procesadores}:

	\begin{itemize}
		\item \textbf{\underline{AMD \underline{Turion} 64 X2 Dual-Core}}\\
		Versión de bajo consumo del procesador AMD Athlon 64 destinada a portátiles. Constituye la respuesta comercial de AMD a la plataforma Centrino de Intel. Los modelos disponibles son:\vspace*{0.3cm}

		\item Lancaster (90 nm)
			\begin{itemize}
			\item Caché L2: 512 o 1024 KB.
			\item Socket 754, HyperTransport (800 MHz, HT800).
			\item Lanzamiento: 10 de marzo, 2005.
			\item Frecuencias de reloj: hasta 2400 MHz.
			\end{itemize}

		\item Richmond (65nm y 90nm)
			\begin{itemize}
			\item Como los Lancaster, salvo que se añade tecnología de virtualización AMD-V.
			\end{itemize}
	\end{itemize}
\end{itemize}
}

