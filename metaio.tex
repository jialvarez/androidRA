\subsection*{Metaio}
\frame
{
\frametitle{Metaio}
\begin{itemize}
 \item Fundado en 2003 en Munich por Thomas Alt y Peter Meier
 \item Se estructura en \textit{canales}
 \item Ofrecen un conjunto de productos:
 \begin{itemize}
   \item \textbf{metaio SDK + metaio Cloud:} SDK de desarrollo para metaio con cuenta de acceso a Cloud. 
   \item \textbf{metaio Creator + metaio Cloud:} aplicación de escritorio para crear AR channels y visualizarlo en junaio.
   \item \textbf{junaio:} navegador de realidad aumentada.
 \end{itemize}
 \item Los canales pueden gestionarse online: \\
   \url{http://dev.junaio.com/index/mychannels}
 \item Disponible para Android, iOS y Windows
\end{itemize}
}

\frame
{
\frametitle{Metaio: ventajas e inconvenientes}
\begin{itemize}
\item \textbf{Ventajas:}
  \begin{itemize}
   \item Posibilidad de reconocimiento en la nube
   \item Posibilidad de montar tu propia servidor de recursos
   \item SDK muy sencillo y bien documentado
   \item Buen soporte orientado a comunidad de desarrolladores
  \end{itemize}

\item \textbf{Inconvenientes:}
  \begin{itemize}
   \item Pequeño lag a veces
   \item Eliminar la marca de agua es caro
   \item No es libre
   \item La plataforma web es demasiado compleja 
  \end{itemize}

\end{itemize}
}

\frame
{
\frametitle{Metaio: recursos}
\begin{itemize}
\item \textbf{Planes de precios:} \url{http://www.metaio.com/pricing/software-and-licensing/}
\item \textbf{Descarga demo metaio Creator:} \url{http://dev.metaio.com/creator/}
\item \textbf{Tutoriales:} \url{http://dev.metaio.com/sdk/tutorials/hello-world/}
\item \textbf{Channels manager:} \\\url{http://dev.junaio.com/index/mychannels}
\end{itemize}
}

