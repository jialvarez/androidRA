\section*{Sección 3}
\frame
{
\frametitle{}
\begin{center}
\begin{huge}
\hspace*{1cm}Un baile de tecnologías\newline\newline
\end{huge}
\includegraphics[width=0.4\textwidth]{imgs/tecnologias.jpg}
\end{center}
}

\section{Un baile de tecnologías}
\frame
{
\frametitle{Tecnologías empleadas por los principales competidores}

\begin{itemize}
 \item \textbf{Tecnologías empleadas por INTEL:}

	\begin{itemize}
	 \item Hyper-Threading
	 \item Intel 64 Architecture
	 \item Bit de desactivación de ejecución y seguridad
	\end{itemize}

\item \textbf{Tecnologías empleadas por AMD:}

	\begin{itemize}
	 \item HyperTransport
	 \item AMD 64
	 \item Bit NX
	\end{itemize}

\end{itemize}
}

\frame
{
\frametitle{Tecnologías empleadas por INTEL}
\begin{itemize}
 \item \textbf{Hyper-Threading:}
 \begin{itemize}
  \item Dos formas de brindar más potencia informática:
	\begin{enumerate}
	 \item Aumentar la velocidad del reloj.
	 \item Realizar más trabajo en \textbf{cada ciclo de reloj}.
	\end{enumerate}
  \item Un procesador \textit{compatible} con la tecnología Hyper-Threading se presenta a sí mismo ante las aplicaciones y los S.O. como dos \textbf{procesadores virtuales}.
  \item El procesador puede entonces trabajar en \textbf{dos conjuntos de tareas} a la vez, utilizar los recursos que de otro modo estarían inactivos y realizar más trabajo en la misma cantidad de tiempo.
  \item En los \textbf{PC de escritorio:}
	\begin{itemize}
	 \item La tecnología HT aprovecha la capacidad de subprocesos múltiples integrada en WinXP y en muchas aplicaciones. El software con subprocesos múltiples divide su carga de trabajo en procesos y subprocesos que se pueden programar y enviar de forma independiente. Es parecido a un sistema multiprocesador pero con \underline{un único procesador}.
	\end{itemize}
 \end{itemize}
\end{itemize}
}

\frame
{
\frametitle{Tecnologías empleadas por INTEL}
\begin{itemize}
  \item En los \textbf{servidores:}
	\begin{itemize}
	 \item La tecnología HT permite el paralelismo a nivel de subprocesos al duplicar el estado arquitectónico de cada procesador a la vez que se comparte un conjunto de recursos de ejecución del procesador. Cuando programa subprocesos, el SO considera los dos estados arquitectónicos claramente determinados como procesadores ``lógicos'' separados
	\end{itemize}
 
\begin{center}
  \includegraphics[width=0.6\textwidth]{imgs/ht.png}
\end{center}
\end{itemize}
}

\frame
{
\frametitle{Tecnologías empleadas por INTEL}
\begin{itemize}
 \item \textbf{Intel 64:}

	\begin{itemize}
		\item La arquitectura \textbf{Intel 64} proporciona \textbf{computación de 64 bits} cuando se combina con \underline{software que la soporte}. 
		\item Mejora el rendimiento permitiendo a los sistemas direccionar\textbf{ más de 4 gigabytes} tanto de \textbf{memoria virtual} como \textbf{física}.
 	\end{itemize}

 \item \textbf{Bit de desactivación de ejecución y seguridad:}

	\begin{itemize}
		\item Previene ciertos tipos de ataques de \textbf{desbordamiento de buffer} cuando se combina con un sistema operativo compatible.
		\item Permite que el procesador clasifique \textbf{áreas de la memoria} en función de dónde se puede ejecutar el código de las aplicaciones.\newline
	\end{itemize}

\begin{center}
\scriptsize \fbox{
\begin{minipage}{11cm}
Si un gusano intenta \textbf{insertar código en el buffer}, el procesador desactiva la ejecución del código, lo cual evita el daño y la propagación del gusano.
\end{minipage}
}
\end{center}
\end{itemize}
}

\frame
{
\frametitle{Tecnologías empleadas por AMD}
\begin{itemize}
 \item \textbf{HyperTransport:}

	\begin{itemize}
		\item Tecnología que induce en una \textbf{mejora de las prestaciones} del sistema, diseñada para incrementar las mismas mediante la \textbf{eliminación de cuellos de botella en la E/S}, lo cual mejora notablemente el ancho de banda y reduce la \textit{latencia}. 

		\item Las mejoras se centran en cuatro apartados:

		\begin{itemize}
		 \item[1)] \textbf{FSB del procesador:} Sustituyendo el FSB por unas conexiones de E/S basadas en la tecnología HyperTransport se consigue una \textbf{extensión del ancho de banda} desde los 2,1GB/s hasta los 6,4GB/s.

		 \item[2)] \textbf{Interfaz de memoria:} Cuando ocurre un fallo en la caché, el procesador debe traer información de memoria principal. En Northbridge/Southbridge, las transacciones de memoria pasan por el chip Northbridge, creando latencias adicionales. Para resolver este cuello de botella, AMD incorpora el controlador de memoria en su 8ª generación de procesadores. 

		\end{itemize}
	\end{itemize}
\end{itemize}
}

\frame
{
\frametitle{Tecnologías empleadas por AMD}
\begin{itemize}
 \item \textbf{HyperTransport:}

	\begin{itemize}
		\item Las mejoras se centran en cuatro apartados:

		\begin{itemize}
		 \item[3)] \textbf{Interconexión chip a chip:} La \textbf{integración simultánea} de las tecnologías de alta velocidad como AGP-8x, Gigabit Ethernet, PCI-X, etc. elimina virtualmente los cuellos de botella en la E/S.

		 \item[4)] \textbf{Capacidades de expansión de E/S hacia la industria de buses de alta velocidad:} La arquitectura Northbridge/Southbridge no está preparada para soportar más de dos \textit{núcleos lógicos}, ya que la funcionalidad debería fijarse a una interfaz existente, y un bus actual no tendría suficiente ancho de banda para soportar tecnologías de alta velocidad.
		\end{itemize}
	\end{itemize}
\end{itemize}
}

\frame
{
\frametitle{Tecnologías empleadas por AMD}
\begin{itemize}
 \item \textbf{HyperTransport:}

	\begin{center}
	\includegraphics[width=0.8\textwidth]{imgs/htransport.png}
	\end{center}

\end{itemize}
}

\frame
{
\frametitle{Tecnologías empleadas por AMD}
\begin{itemize}
 \item \textbf{AMD64:} es una arquitectura basada en la \textbf{extensión} del conjunto de instrucciones x86 para manejar direcciones de 64 bits. Además, contempla mejoras adicionales como \textbf{duplicar el número y el tamaño} de los registros de uso general y de instrucciones SSE.

 \item \textbf{Bit NX:} el bit NX es una característica del procesador que permite al SO prohibir la ejecución del código en ciertas áreas de datos. 
\end{itemize}
}
